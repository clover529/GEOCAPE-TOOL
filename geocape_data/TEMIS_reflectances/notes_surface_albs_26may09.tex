Note from Piet Stammes:

Dear Kelly,

Sorry that I missed your email due to travel and holiday.

The GOME-1 surface albedo database is online and freely available via the TEMIS website www.temis.nl.
Click via Utilities to LER database, or go directly to:

http://www.temis.nl/data/ler.html

Please refer to the Koelemeijer et al. paper and give an acknowledgement to TEMIS.

Success,

regards,
Piet
ear All:

The surface albedo database I mentioned is actually from almost 20 years of TOMS data. I got it from Changwoo Ahn and Omar Torres. The data include monthly mean surface albedo at 4 wavelengths 331, 340, 360, 380 on 1 degree x 1degree grid cells. You can find the data at:
http://www.cfa.harvard.edu/~xliu/toms_sfcalb_clima.tar
which contains four files, each for one wavelength. There are some header about how to read the data in each file.

Since there are 4 wavelengths, I suggest using the surface albedo at 331 nm since it is closest to the ozone retrieval spectral region (270-340 nm) in the UV.

Please let me or Changwoo/Omar know if you have any questions about the data

Best regards.

Xiong


Eldering, Annmarie wrote:
> Hi all -
>
> We have planned to have a telecon on Thursday at 10am Pacific, 1pm Eastern.
> Our secretary, Darlene Padgett is setting up the telecon/meetingplace, details will be sent tomorrow.
>
> She is setting up a meetingplace web site so we can share document easily.
>
> Kelly has sent out some cross section info, I will get some of my material up tomorrow.
>
> Please send in agenda items.
>
> My current list:
>  - status of preparation for RT calcs (code and inputs)
>  - atmospheres from Ken and Annmarie
>  - surface properties from Kelly
>  - cross sections from Kelly and interface to lidort
>  - agree on high level timeline and milestones for next six months, and detailed milestones for a couple of months
>
>
> Actions from last meeting:
>
> I will post some atmospheres and surface properties in the infrared, and describe what we do to run LBLRTM.
> I will email Rob Spurr and Xiong to inlcude them
> Joanna will talk to Rob to get LIDORT running
> Kelly will share the GOME data for surface properties, suggest spectral grids to use for sampling, and share information about the cross sections that they use.
>
> Thanks,
>
> Annmarie
>
>
> On 5/15/09 8:38 AM, "Kelly Chance" <kchance@cfa.harvard.edu> wrote:
>
>     Dear Colleagues,
>
>     Please see attached. The description is in NASA-O3Z.inf. I will try to
>     upload it to the wiki before I leave for Europe tomorrow, but you have
>     it in any case.
>
>     Best, kelly
>
