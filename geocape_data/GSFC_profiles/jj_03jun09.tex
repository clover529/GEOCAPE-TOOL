Hi Rob,

That's easy enough. At some point, I think it would be better for the rest of us (Arlindo included) to have a clean fortran interface to LIDORT with data structures as the inputs and outputs. I think we should work together on defining that interface. I had already put something together for clouds, but it needs to be more generic than that. In any case, that is something for the future.

I'll send you a test file shortly. I will not include the surface albedo. We can iterate on that profile as needed.

Joanna

ROBERT SPURR wrote:
> Hi Joanna,
>  I would prefer ASCII, I am not really equipped to handle anything too fancy.
>  Best   Rob
>  Robert J. D. Spurr, MA PhD FRMetS
> Director, RT SOLUTIONS Inc.
> 9 Channing Street, Cambridge MA 02138, USA
> Tel : +1 617 492 1183
> Fax : +1 617 497 9390
> email : rtsolutions@verizon.net
>
> European affiliation and contact:
> Belgian Institute for Space Aeronomy (BIRA-IASB)
> Avenue Circulaire 3, B-1180 Brussels, Belgium
> email: robs@oma.be
>
>
> ------------------------------------------------------------------------
> *From:* Joanna Joiner <joanna.joiner@nasa.gov>
> *To:* ROBERT SPURR <rtsolutions@verizon.net>; Xiong Liu <xliu@umbc.edu>
> *Sent:* Tuesday, June 2, 2009 3:15:50 PM
> *Subject:* format of data
>
> Hi Rob, Xiong,
>
> I have the profiles. I start by sending you a single profile. How would you like to receive it?
>
> Currently there are two ASCII files, one for surface skin temp and one for everything else. I can write some code to get that into a data structure for you or put it into one netcdf file and supply you a reader. Just let me know what would be the most appropriate.
>
> Thanks,
> Joanna
>